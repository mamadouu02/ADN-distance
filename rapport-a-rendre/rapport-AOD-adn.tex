\documentclass[10pt,a4paper]{article}

\usepackage[utf8]{inputenc}
% \usepackage[T1]{fontenc}
\usepackage{enumerate}
\usepackage{amssymb}

%%%% POUR FAIRE TENIR SUR UNE PAGE RECTO-VERSO.....
\textwidth 18.5cm
\oddsidemargin -1.75cm
\evensidemargin -1.75cm
\textheight 28.0cm
\topmargin -3.0cm

%   \textwidth 18cm
   %\oddsidemargin -1.5cm
   %\evensidemargin -1.5cm
   %\textheight 26.0cm
   %\topmargin -2.0cm
 


\begin{document}

\thispagestyle{empty}

\noindent\centerline{\bf\large Questionnaire  TP AOD 2023-2024 à compléter et rendre sur teide  } \\
Binôme 
(NOM\textsubscript{1} Prénom\textsubscript{1} --
 NOM\textsubscript{2} Prénom\textsubscript{2})
\,: BOUDAHMANE Ahmed -- THIONGANE Mamadou

\section{Préambule (1 point)}
Le programme récursif avec mémoisation alloue une mémoire de taille $N.M$.
Il génère une erreur d'exécution sur le test 5 (ci-dessous). Pourquoi ?

\begin{verbatim}
distanceEdition-recmemo    GCA_024498555.1_ASM2449855v1_genomic.fna 77328790 20236404   \
                           GCF_000001735.4_TAIR10.1_genomic.fna 30808129 19944517 
\end{verbatim}

\noindent Réponse : Les séquences étant très longues sur le test 5, la mémoire demandée par le programme récursif pour l'allocation d'une mémoire de taille $N \times M$ et pour la pile d'appels récursifs est trop importante.

\medskip

%%%%%%%%%%%%%%%%%%%
{\noindent\bf{Important}.} Dans toute la suite, on demande des programmes qui allouent un espace mémoire $O(N+M)$.

\section{Programme itératif en espace mémoire $O(N+M)$ (5 points)}
{\em Expliquer très brièvement (2 à 5 lignes max) le principe de votre code, la mémoire utilisée, le sens de parcours des tableaux.}

\medskip

À partir de $\phi(M,N)$, on calcule $\left( \phi(i,j) \right)_{0 \leqslant i \leqslant M, 0 \leqslant j \leqslant N}$ itérativement par lignes. Le sens de parcours s'effectue de droite à gauche en remontant par lignes. Les $\phi(i,j)$ sont stockés par lignes en utilisant un tableau de taille $N+1$.

\medskip

Analyse du coût théorique de ce  programme en fonction de $N$ et $M$  en notation $\Theta(...)$ 
\begin{enumerate}
  \item place mémoire allouée (ne pas compter les 2 séquences $X$ et $Y$ en mémoire via {\tt mmap}) : $\Theta(N)$
  \item travail (nombre d'opérations) : $\Theta(NM)$
  \item nombre de défauts de cache obligatoires (sur modèle CO, y compris sur $X$ et $Y$): $\Theta \left( \frac{NM}{L} \right)$ sur le tableau, $\Theta \left( \frac{M}{L} \right)$ sur $X$ et $\Theta \left( \frac{N}{L} \right)$ sur $Y$
  \item nombre de défauts de cache si $Z \ll \min(N,M)$ : $\Theta \left( \frac{NM}{L} \right)$ sur le tableau, $\Theta \left( \frac{M}{L} \right)$ sur $X$ et $\Theta \left( \frac{NM}{L} \right)$ sur $Y$
\end{enumerate}

%%%%%%%%%%%%%%%%%%%
\section{Programme cache aware  (3 points)}
{\em Expliquer très brièvement (2 à 5 lignes max) le principe de votre code, la mémoire utilisée, le sens de parcours des tableaux.}

\medskip

À partir de $\phi(M,N)$, on calcule $\left( \phi(i,j) \right)_{0 \leqslant i \leqslant M, 0 \leqslant j \leqslant N}$ itérativement par blocs de taille $K \times K$. Le sens de parcours s'effectue de droite à gauche en remontant par blocs, chaque bloc étant parcouru par lignes. Les $\phi(i,j)$ sont stockés par lignes en utilisant un tableau de taille $N+1$ et les $\phi(i,j)$ d'un bloc sont stockés par colonnes en utilisant un tableau de taille $K = \sqrt{Z}$.

\medskip

Analyse du coût théorique de ce programme en fonction de $N$ et $M$  en notation $\Theta(...)$ )
\begin{enumerate}
  \item place mémoire allouée (ne pas compter les 2 séquences $X$ et $Y$ en mémoire via {\tt mmap}) : $\Theta(N + \sqrt{Z})$
  \item travail (nombre d'opérations) : $\Theta(NM)$
  \item nombre de défauts de cache obligatoires (sur modèle CO, y compris sur $X$ et $Y$): $\Theta \left( \frac{NM}{L} \right)$ sur le tableau, $\Theta \left( \frac{NM}{\sqrt{Z} L} \right)$ sur $X$ et $\Theta \left( \frac{NM}{\sqrt{Z} L} \right)$ sur $Y$
  \item nombre de défauts de cache si $Z \ll \min(N,M)$ : $\Theta \left( \frac{NM}{L} \right)$ sur le tableau, $\Theta \left( \frac{NM}{\sqrt{Z} L} \right)$ sur $X$ et $\Theta \left( \frac{NM}{\sqrt{Z} L} \right)$ sur $Y$
\end{enumerate}

%%%%%%%%%%%%%%%%%%%
\section{Programme cache oblivious  (3 points)}
{\em Expliquer très brièvement (2 à 5 lignes max) le principe de votre code, la mémoire utilisée, le sens de parcours des tableaux.}

\medskip

À partir de $\phi(M,N)$, on calcule $\left( \phi(i,j) \right)_{0 \leqslant i \leqslant M, 0 \leqslant j \leqslant N}$ récursivement par blocs de taille $S \times N$, où $S$ est le seuil de récursivité choisi de sorte que les données de $X$ accédées dans un bloc tiennent dans le cache. Le sens de parcours s'effectue en remontant par blocs, chaque bloc étant parcouru en remontant par colonnes, de droite à gauche. Les $\phi(i,j)$ sont stockés par lignes en utilisant un tableau de taille $N+1$ et les $\phi(i,j)$ d'un bloc sont stockés par colonnes en utilisant un tableau de taille $S$.

\medskip

Analyse du coût théorique de ce  programme en fonction de $N$ et $M$  en notation $\Theta(...)$ )
\begin{enumerate}
  \item place mémoire allouée (ne pas compter les 2 séquences $X$ et $Y$ en mémoire via {\tt mmap}) : $\Theta(N + S)$
  \item travail (nombre d'opérations) : $\Theta(NM)$
  \item nombre de défauts de cache obligatoires (sur modèle CO, y compris sur $X$ et $Y$): $\Theta \left( \frac{NM}{L} \right)$ sur le tableau, $\Theta \left( \frac{M}{L} \right)$ sur $X$ et $\Theta \left( \frac{MN}{SL} \right)$ sur $Y$
  \item nombre de défauts de cache si $Z \ll \min(N,M)$ : $\Theta \left( \frac{NM}{L} \right)$ sur le tableau, $\Theta \left( \frac{MN}{L} \right)$ sur $X$ et $\Theta \left( \frac{NM}{SL} \right)$ sur $Y$
\end{enumerate}

\section{Réglage du seuil d'arrêt récursif du programme cache oblivious  (1 point)} 
Comment faites-vous sur une machine donnée pour choisir ce seuil d'arrêt? Quelle valeur avez vous choisi pour les
PC de l'Ensimag? (2 à 3 lignes)

\medskip

On a raisonné par dichotomie pour trouver le meilleur seuil : on relève le nombre de défauts de cache fournies par \texttt{valgrind} ainsi que le temps d'éxecution du programme sur plusieurs tests et si on observe un amélioration des performances entre deux tests on conserve la dernière valeur. On a donc choisi $S=125$.

%%%%%%%%%%%%%%%%%%%
\section{Expérimentation (7 points)}

Description de la machine d'expérimentation:  \\
Processeur: Intel Core i5-7500 CPU @ 3.40GHz × 4 --
Mémoire: 32.0 GiB --
Système: Dell Inc. Precision Tower 3420

\subsection{(3 points) Avec {\tt 
	valgrind --tool=cachegrind --D1=4096,4,64
}} 
\begin{verbatim}
     distanceEdition ba52_recent_omicron.fasta 153 N wuhan_hu_1.fasta 116 M 
\end{verbatim}
en prenant pour $N$ et $M$ les valeurs dans le tableau ci-dessous.

\medskip

Les paramètres du cache LL de second niveau sont : \texttt{6291456 B, 64 B, 12-way associative}

\medskip

\begin{center}
    {\footnotesize
    \begin{tabular}{|r|r||r|r|r||r|r|r||}
    \hline
     \multicolumn{2}{|c||}{ } 
    & \multicolumn{3}{c||}{récursif mémo}
    & \multicolumn{3}{c||}{itératif}
    \\ \hline
    $N$ & $M$ 
    & \#Irefs & \#Drefs & \#D1miss % recursif memoisation
    & \#Irefs & \#Drefs & \#D1miss % itératif
    \\ \hline
    \hline
    1000 & 1000 
    & 217,200,815 & 122,122,238 & 4,925,369  % recursif memoisation
    & 103,321,402 & 42,541,770 & 148,791  % itératif
    \\ \hline
    2000 & 1000 
    & 433,378,209 & 243,401,470 & 11,022,819  % recursif memoisation
    & 206,058,363 & 85,001,754 & 292,606  % itératif
    \\ \hline
    4000 & 1000 
    & 867,150,307 & 487,365,726 & 23,222,782  % recursif memoisation
    & 411,233,568 & 169,921,634 & 580,249  % itératif
    \\ \hline
    2000 & 2000 
    & 867,141,905 & 487,888,421 & 19,896,557  % recursif memoisation
    & 412,481,161 & 169,899,591 & 574,302  % itératif
    \\ \hline
    4000 & 4000 
    & 3,465,864,618 & 1,950,548,616 & 80,000,438  % recursif memoisation
    & 1,648,944,597 & 679,272,871 & 2,262,913  % itératif
    \\ \hline
    6000 & 6000 
    & 7,796,325,080 & 4,387,984,427 & 180,346,527  % recursif memoisation
    & 3,709,599,857 & 1,528,189,672 & 5,076,466  % itératif
    \\ \hline
    8000 & 8000 
    & 13,857,954,627 & 7,799,947,998 & 321,302,450  % recursif memoisation
    & 6,594,444,508 & 2,716,649,992 & 9,019,928  % itératif
    \\ \hline
    \hline
    \end{tabular}
    }

    \medskip
    
    {\footnotesize
    \begin{tabular}{|r|r||r|r|r||r|r|r||}
    \hline
     \multicolumn{2}{|c||}{ } 
    & \multicolumn{3}{c||}{cache aware}
    & \multicolumn{3}{c||}{cache oblivious}
    \\ \hline
    $N$ & $M$ 
    & \#Irefs & \#Drefs & \#D1miss % cache aware
    & \#Irefs & \#Drefs & \#D1miss % cache oblivious
    \\ \hline
    \hline
    1000 & 1000 
    & 118,750,875 & 51,792,350 & 7,362  % cache aware
    & 139,073,200 & 66,379,265 & 6,525  % cache oblivious
    \\ \hline
    2000 & 1000 
    & 236,903,406 & 103,495,478 & 9,843  % cache aware
    & 277,535,016 & 132,659,677 & 8,268  % cache oblivious
    \\ \hline
    4000 & 1000 
    & 472,909,239 & 206,901,349 & 14,681  % cache aware
    & 554,163,276 & 265,221,667 & 13,072  % cache oblivious
    \\ \hline
    2000 & 2000 
    & 474,160,260 & 206,882,587 & 14,779  % cache aware
    & 555,413,924 & 265,201,830 & 11,643  % cache oblivious
    \\ \hline
    4000 & 4000 
    & 1,895,444,821 & 827,084,387 & 52,392  % cache aware
    & 2,220,523,172 & 1,060,384,536 & 39,438  % cache oblivious
    \\ \hline
    6000 & 6000 
    & 4,264,063,812 & 1,860,675,440 & 92,168  % cache aware
    & 4,991,734,938 & 2,386,191,401 & 89,535  % cache oblivious
    \\ \hline
    8000 & 8000 
    & 7,580,302,267 & 3,307,820,288 & 160,715  % cache aware
    & 8,880,432,681 & 4,240,893,529 & 174,261 % cache oblivious
    \\ \hline
    \hline
    \end{tabular}
    }
\end{center}

\paragraph{Important: analyse expérimentale:}
ces mesures expérimentales sont-elles en accord avec les coûts analysés théoriquement (justifier) ? 
Quel algorithme se comporte le mieux avec valgrind et 
les paramètres proposés, pourquoi ?

\medskip

Pour la première ligne, le produit $N\times M$ est de l'ordre de $10^6$. Comme $L = 64$, la valeur de $\frac{NM}{L}$ est de l'ordre de 15000. Seuls les programmes cache-aware et cache-oblivious respectent cet ordre de grandeur. De plus, les nombres de défauts de cache sont bien plus faibles pour les programmes cache-aware et cache-oblivious que ceux des programmes récursif et itératif. L'algorithme qui se comporte le mieux avec \texttt{valgrind} pour les petites valeurs de $N$ et $M$ est le programme cache-oblivious mais le programme cache-aware génère moins de défauts de cache pour les plus grandes valeurs. En effet, les programmes cache-aware et cache-oblivious utilisent des blocs de sorte que les valeurs accédées dans un bloc tiennent dans le entièrement dans le cache complètement pour ainsi réduire le nombre de défauts de cache.

\subsection{(3 points) Sans valgrind, par exécution de la commande :}
{\tt \begin{tabular}{llll}
distanceEdition & GCA\_024498555.1\_ASM2449855v1\_genomic.fna & 77328790 & M \\
                & GCF\_000001735.4\_TAIR10.1\_genomic.fna     & 30808129 & N
\end{tabular}}

\medskip

On mesure respectivement le temps écoulé, le temps CPU et l'énergie consommée avec {\tt getimeofday}, {\tt getrusage} et {\tt get\_energy\_uj\_counter}.

\medskip

L'énergie consommée sur le processeur peut être estimée en regardant le compteur RAPL d'énergie (en microJoule)
pour chaque core avant et après l'exécution et en faisant la différence.
Le compteur du core $K$ est dans le fichier 
\verb+ /sys/class/powercap/intel-rapl/intel-rapl:K/energy_uj +\\
Par exemple, pour le c{\oe}ur 0:
\verb+ /sys/class/powercap/intel-rapl/intel-rapl:0/energy_uj +

Nota bene: pour avoir un résultat fiable/reproductible (si variailité), 
il est préférable de faire chaque mesure 5 fois et de reporter l'intervalle
de confiance [min, moyenne, max]. 

\begin{center}
    \begin{tabular}{|r|r||r|r|r||}
    \hline
     \multicolumn{2}{|c||}{ } 
    & \multicolumn{3}{c||}{itératif}
    \\ \hline
    $N$ & $M$ 
    & temps   & temps & energie       % itératif
    \\
    & 
    & cpu     & écoulé&               % itératif
    \\ \hline
    \hline
    10000 & 10000 
    & [1.26, 1.27, 1.28] & [1.27, 1.27, 1.28] & [5.01E-06, 5.58E-06, 5.99E-06]  % itératif
    \\ \hline
    20000 & 20000 
    & [5.07, 5.15, 5.32] & [5.07, 5.15, 5.33] & [2.01E-05, 2.40E-05, 2.80E-05]  % itératif
    \\ \hline
    30000 & 30000 
    & [11.40, 11.51, 11.61] & [11.41, 11.55, 11.68] & [4.59E-05, 5.16E-05, 5.98E-05]  % itératif
    \\ \hline
    40000 & 40000 
    & [20.41, 20.51, 20.63] & [20.41, 20.52, 20.64] & [8.76E-05, 9.15E-05, 1.00E-04]  % itératif
    \\ \hline
    \hline
    \end{tabular}
    
    \medskip
    
    \begin{tabular}{|r|r||r|r|r||}
    \hline
     \multicolumn{2}{|c||}{ } 
    & \multicolumn{3}{c||}{cache aware}
    \\ \hline
    $N$ & $M$ 
    & temps   & temps & energie       % cache aware
    \\
    & 
    & cpu     & écoulé&               % cache aware
    \\ \hline
    \hline
    10000 & 10000 
    & [0.94, 0.95, 0.97] & [0.94, 0.95, 0.97] & [3.81E-06, 4.64E-06, 6.18E-06]  % cache aware
    \\ \hline
    20000 & 20000 
    & [3.77, 3.79, 3.81] & [3.77, 3.79, 3.81] & [1.55E-05, 1.74E-05, 2.39E-05]  % cache aware
    \\ \hline
    30000 & 30000 
    & [8.55, 8.59, 8.61] & [8.55, 8.59, 8.61] & [3.83E-05, 3.91E-05, 4.04E-05]  % cache aware
    \\ \hline
    40000 & 40000 
    & [15.25, 15.27, 15.29] & [15.26, 15.27, 15.29] & [6.86E-05, 6.94E-05, 7.10E-05]  % cache aware
    \\ \hline
    \hline
    \end{tabular}
    
    \medskip
    
    \begin{tabular}{|r|r||r|r|r||}
    \hline
     \multicolumn{2}{|c||}{ } 
    & \multicolumn{3}{c||}{cache oblivious}
    \\ \hline
    $N$ & $M$ 
    & temps   & temps & energie       % cache oblivious
    \\
    & 
    & cpu     & écoulé&               % cache oblivious
    \\ \hline
    \hline
    10000 & 10000 
    & [1.27, 1.30, 1.38] & [1.27, 1.31, 1.39] & [5.11E-06, 6.88E-06, 9.99E-06]  % cache oblivious
    \\ \hline
    20000 & 20000 
    & [5.10, 5.17, 5.39] & [5.10, 5.19, 5.40] & [2.08E-05, 2.33E-05, 2.79E-05]  % cache oblivious
    \\ \hline
    30000 & 30000 
    & [11.49, 11.55, 11.67] & [11.50, 11.58, 11.71] & [4.96E-05, 5.28E-05, 5.81E-05]  % cache oblivious
    \\ \hline
    40000 & 40000 
    & [20.52, 20.62, 20.72] & [20.55, 20.67, 20.81] & [9.23E-05, 9.78E-05, 1.06E-04]  % cache oblivious
    \\ \hline
    \hline
    \end{tabular}
\end{center}

\paragraph{Important: analyse expérimentale:} 
ces mesures expérimentales sont-elles en accord avec les coûts analysés théoriquement (justifier) ? 
Quel algorithme se comporte le mieux avec valgrind et 
les paramètres proposés, pourquoi ?

\medskip

L'algorithme qui se comporte le mieux avec \texttt{valgrind} est le programme cache-aware qui est meilleur en temps et en énergie. Plusieurs raisons peuvent justifier cela : le programme itératif est plus lent à cause de son nombre de défauts de cache très important par rapport au programme cache-aware et pour le programme cache-oblivious, qui génère autant de défauts de cache que le programme cache-aware, est possiblement plus lent que ce dernier à cause d'un grand nombre d'appel récursifs.

\subsection{(1 point) Extrapolation: estimation de la durée et de l'énergie pour la commande :}
{\tt \begin{tabular}{llll}
distanceEdition & GCA\_024498555.1\_ASM2449855v1\_genomic.fna & 77328790 & 20236404  \\
                & GCF\_000001735.4\_TAIR10.1\_genomic.fna     & 30808129 & 19944517 
\end{tabular}
}

\medskip

À partir des résultats précédents, le programme cache-aware est le plus performant pour la commande ci-dessus (test 5). Les ressources pour l'exécution, obtenues par régression polynomiale avec $N=M=20000000$ et un coefficient de corrélation $R^2=0.99$, seraient environ: 
\begin{itemize}
\item Temps cpu (en s) : 3716000
\item Energie  (en kWh) : 12.58
\end{itemize}

Question subsidiaire: comment feriez-vous pour avoir un programme s'exécutant en moins de 1 minute ? 
{\em Donner le principe en moins d'une ligne, même 1 mot précis suffit! } 

\medskip

Par parallélisme.

\end{document}
